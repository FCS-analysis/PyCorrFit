\subsection{Included model functions}
This is an overview of all the model functions that are currently\footnote{\today} implemented in PyCorrFit. To each model a unique model ID is assigned by PyCorrFit. The following information is also accessible from within PyCorrFit using the \textbf{Page info} tool.

\subsubsection{Confocal FCS}
The detection volume is a confocal volume with the structural parameter 
\begin{align}
\mathit{SP}= \frac{z_0}{r_0}
\end{align}
where $r_0$ is its lateral and $z_0$ its axial (in case of 3D diffusion) extension. The diffusion coefficient is calculated from the diffusion time $\tau_\mathrm{diff}$ using
\begin{align}
D = \frac{1}{4 \tau_\mathrm{diff}} \left( \frac{z_0}{\mathit{SP}} \right)^2 = \frac{r_0^2}{4 \tau_\mathrm{diff}}.
\end{align}
\\


% 2D diffusion
\noindent \begin{tabular}{lp{.7\textwidth}}
Name & \textbf{2D (Gauß)} \\ 
ID & \textbf{6001} \\ 
Descr. &  Two-dimensional diffusion with a Gaussian laser profile. \\ 
\end{tabular}
\begin{align}
G(\tau) = A_0 + \frac{1}{N} \frac{1}{(1+\tau/\tau_\mathrm{diff}) )}
\end{align} 
\begin{center}
\begin{tabular}{ll}
$A_0$ & Offset \\ 
$N$ & Effective number of particles in confocal area \\ 
$\tau_\mathrm{diff}$ &   Characteristic residence time in confocal area \\
\end{tabular} \\
\end{center}
\vspace{2em}


% 2D diffusion + triplett
\noindent \begin{tabular}{lp{.7\textwidth}}
Name & \textbf{2D+T (Gauß)} \\ 
ID & \textbf{6002} \\ 
Descr. &  Two-dimensional diffusion with a Gaussian laser profile, including a triplet component. \\ 
\end{tabular}
\begin{align}
G(\tau) = A_0 + \frac{1}{N} \frac{1}{(1+\tau/\tau_\mathrm{diff}))}  \left(1 + \frac{T e^{-\tau/\tau_\mathrm{trip}}}{1-T}  \right)
\end{align} 
\begin{center}
\begin{tabular}{ll}
$A_0$ & Offset \\ 
$N$ & Effective number of particles in confocal area \\ 
$\tau_\mathrm{diff}$ &  Characteristic residence time in confocal area \\ 
$T$ &  Fraction of particles in triplet (non-fluorescent) state\\ 
$\tau_\mathrm{trip}$ &  Characteristic residence time in triplet state \\ 
\end{tabular}
\end{center}
\vspace{2em}


% 3D diffusion
\noindent \begin{tabular}{lp{.7\textwidth}}
Name & \textbf{3D (Gauß)} \\ 
ID & \textbf{6012} \\ 
Descr. &  3D free diffusion with a Gaussian laser profile (eliptical). \\ 
\end{tabular}
\begin{align}
G(\tau) = A_0 + \frac{1}{N} \frac{1}{(1+\tau/\tau_\mathrm{diff}))} \frac{1}{\sqrt{1+\tau/(\mathit{SP}^2 \tau_\mathrm{diff})}}
\end{align} 
\begin{center}
\begin{tabular}{ll}
$A_0$ & Offset \\ 
$N$ & Effective number of particles in confocal volume \\ 
$\tau_\mathrm{diff}$ &  Characteristic residence time in confocal volume \\ 
$\mathit{SP}$ & Structural parameter, describes elongation of the confocal volume \\
\end{tabular}
\end{center}
\vspace{2em}


% 3D diffusion + triplet
\noindent \begin{tabular}{lp{.7\textwidth}}
Name & \textbf{3D+T (Gauß)} \\ 
ID & \textbf{6011} \\ 
Descr. &  3D free diffusion with a Gaussian laser profile (eliptical). \\ 
\end{tabular}
\begin{align}
G(\tau) = A_0 + \frac{1}{N} \frac{1}{(1+\tau/\tau_\mathrm{diff}))} \frac{1}{\sqrt{1+\tau/(\mathit{SP}^2 \tau_\mathrm{diff})}} \left(1 + \frac{T e^{-\tau/\tau_\mathrm{trip}}}{1-T}  \right)
\end{align} 
\begin{center}
\begin{tabular}{ll}
$A_0$ & Offset \\ 
$N$ & Effective number of particles in confocal volume \\ 
$\tau_\mathrm{diff}$ &  Characteristic residence time in confocal volume \\ 
$\mathit{SP}$ & Structural parameter, describes elongation of the confocal volume \\
$T$ &  Fraction of particles in triplet (non-fluorescent) state\\ 
$\tau_\mathrm{trip}$ &  Characteristic residence time in triplet \\
\end{tabular}
\end{center}
\vspace{2em}


% 2D+2D diffusion + triplett
\noindent \begin{tabular}{lp{.7\textwidth}}
Name & \textbf{2D+2D+T (Gauß)} \\ 
ID & \textbf{6031} \\ 
Descr. &  Two-component, two-dimensional diffusion with a Gaussian laser profile, including a triplet component. \\ 
\end{tabular}
\begin{align}
G(\tau) = A_0 + \frac{1}{N (F + \alpha (1-F))²} \left[ \frac{F}{1+\tau/\tau_1} + \alpha^2 \frac{1-F}{ 1+\tau/\tau_2 } \right] \left(1 + \frac{T e^{-\tau/\tau_\mathrm{trip}}}{1-T}  \right) 
\end{align} 
\begin{center}
\begin{tabular}{ll}
$A_0$ & Offset \\ 
$N$ & Effective number of particles in confocal area ($N = N_1+N_2$) \\ 
$\tau_1$ &  Diffusion time of particle species 1 \\ 
$\tau_2$ &  Diffusion time of particle species 2 \\ 
$F$ & Fraction of molecules of species 1 ($N_1 = F N$) \\
$\alpha$ & Relative molecular brightness of particles 1 and 2 ($ \alpha = q_2/q_1$) \\
$T$ &  Fraction of particles in triplet (non-fluorescent) state\\ 
$\tau_\mathrm{trip}$ &  Characteristic residence time in triplet state \\ 
\end{tabular}
\end{center}
\vspace{2em}


% 3D+2D diffusion + triplett
\begin{samepage}
\noindent \begin{tabular}{lp{.7\textwidth}}
Name & \textbf{3D+2D+T (Gauß)} \\ 
ID & \textbf{6032} \\ 
Descr. &  Two-component, two- and three-dimensional diffusion with a Gaussian laser profile, including a triplet component. \\ 
\end{tabular}
\begin{align}
G(\tau) = A_0 + \frac{1}{N (1 - F + \alpha F)²} \left[ \frac{1-F}{1+\tau/\tau_\mathrm{2D}} + \frac{ \alpha^2 F}{ 1+\tau/\tau_\mathrm{3D} } \frac{1}{\sqrt{1+\tau/(\mathit{SP}^2 \tau_\mathrm{3D})}} \right] \left(1 + \frac{T e^{-\tau/\tau_\mathrm{trip}}}{1-T}  \right) 
\end{align} 
\begin{center}
\begin{tabular}{ll}
$A_0$ & Offset \\ 
$N$ & Effective number of particles in confocal volume ($N = N_\mathrm{2D}+N_\mathrm{3D}$) \\ 
$\tau_\mathrm{2D}$ &  Diffusion time of surface bound particle species \\ 
$\tau_\mathrm{3D}$ &  Diffusion time of freely diffusing particle species \\ 
$F$ & Fraction of molecules of freely diffusing species ($N_\mathrm{3D} = F N$) \\
$\alpha$ & Relative molecular brightness of particle species ($ \alpha = q_\mathrm{3D}/q_\mathrm{2D}$) \\
$\mathit{SP}$ & Structural parameter, describes elongation of the confocal volume \\
$T$ &  Fraction of particles in triplet (non-fluorescent) state\\ 
$\tau_\mathrm{trip}$ &  Characteristic residence time in triplet state \\ 
\end{tabular}
\end{center}
\end{samepage}
\vspace{2em}

\subsubsection{Confocal TIR-FCS}
%G(\tau) = A_0 + \frac{1}{N} \frac{1}{(1+\tau/\tau_\mathrm{diff}) )}

\subsubsection{TIR-FCS with a square shaped lateral detection volume}